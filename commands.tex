\usepackage{expl3}
\usepackage{xparse}

\usepackage{mleftright}
\setlength{\delimitershortfall}{-1sp}

\ExplSyntaxOn

\NewDocumentCommand \I {} { \symup{i} }
\NewDocumentCommand \E {} { \symup{e} }
\NewDocumentCommand \mpi {} { \symup{\pi} }

\NewDocumentCommand \dif {m} { \mathinner{\symup{d} #1} }
\NewDocumentCommand \Dif {m} { \mathinner{\symup{D} #1} }
\NewDocumentCommand \del {m} { \mathinner{\symup{\delta} #1} }
\NewDocumentCommand \Del {m} { \mathinner{\symup{\Delta} #1} }

\let\vaccent=\v % alten Befehl kopieren
\RenewDocumentCommand \v {} % Befehl überschreiben
  {
    \TextOrMath{
      \vaccent % Textmodus
    }{
      \symbf % Mathemodus
    }
  }

\AtBeginDocument{ % wird bei \begin{document} ausgeführt
  % werden sonst wieder von unicode-math überschrieben
  \RenewDocumentCommand \Re {} {\operatorname{Re}}
  \RenewDocumentCommand \Im {} {\operatorname{Im}}
}

\let\ltext=\l
\RenewDocumentCommand \l {}{ \TextOrMath{ \ltext }{ \mleft } }
\let\raccent=\r
\RenewDocumentCommand \r {}{ \TextOrMath{ \raccent }{ \mright } }

\DeclarePairedDelimiter{\abs}{\lvert}{\rvert}
\DeclarePairedDelimiter{\norm}{\lVert}{\rVert}

\DeclarePairedDelimiter{\bra}{\langle}{\rvert}
\DeclarePairedDelimiter{\ket}{\lvert}{\rangle}
% <name> <#arguments> <left> <right> <body>
\DeclarePairedDelimiterX{\braket}[2]{\langle}{\rangle}{#1 \delimsize| #2}


\NewDocumentCommand \OverfullCenter {+m} {
\noindent\makebox[\linewidth]{#1} }

\ExplSyntaxOff
